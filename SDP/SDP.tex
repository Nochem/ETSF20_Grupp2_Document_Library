\documentclass[paper=a4, fontsize=11pt,twoside]{article}

% --------------------------------------------------------------------
% General Page Layout
% --------------------------------------------------------------------
\usepackage[a4paper]{geometry}
\usepackage[parfill]{parskip}
\setlength{\oddsidemargin}{5mm}		% Remove 'twosided' indentation
\setlength{\evensidemargin}{5mm}

% --------------------------------------------------------------------
% Encoding and Language Settings
% --------------------------------------------------------------------
\usepackage[T1]{fontenc}
\usepackage[utf8]{inputenc}			% encoding may need to be changed depending on the system
\usepackage[swedish]{babel}
\usepackage{lipsum}					% Lorem Ipsum

% --------------------------------------------------------------------
% Utilities (colors, links, pictures, ect...)
% --------------------------------------------------------------------
\usepackage{xcolor}
\usepackage{hyperref}
\usepackage{graphicx}
\usepackage{amssymb}
\usepackage{epstopdf}
\usepackage[round]{natbib}
\usepackage{float}
\DeclareGraphicsRule{.tif}{png}{.png}{`convert #1 `dirname #1`/`basename #1 .tif`.png}

% -------------------------------------------------------------------------------------------------------------%
% Title Page / Document Class Definitions (Please Don't Play With This)			%
% -------------------------------------------------------------------------------------------------------------%
																%
% Horizontal rule													%
\newcommand{\HRule}[1]{\rule{\linewidth}{#1}}   							%
																%
% Document Number												%
\newcommand{\documentNumber}[1]{\centering PUSP1742#1 \\[1.0cm]}	 		%
																%
% Document Version													%
\newcommand{\documentVersion}[1]{\centering \small{v.#1} \\[1.0cm]}	 		%
																%
% Title																			%
\makeatletter                           											%
\def\printtitle{%                       											%
    {\centering \@title\par}}												%
\makeatother                                    										%
																%
% Author															%
\makeatletter                           											%
\def\printauthor{%                  											%
    {\centering \large \@author}}               									%
\makeatother														%
																%
\newcommand{\grouptitlepage}[4]{										%
	\title{ 														%
	\documentNumber{#1}											%
	\documentVersion{#2}											%
	\HRule{0.5pt} \\ % Upper rule										%
	\LARGE \textbf{\uppercase{#3}} \\  									%
	\large \textbf{\uppercase{ETSF20 Grupp 2}}							%
	\HRule{2pt} \\ [0.5cm]      	% Lower rule + 0.5cm spacing					%
	\normalsize          		% Todays date								%
	}															%
	\author{#4}													%
	\maketitle														%
	\tableofcontents												%
	\thispagestyle{empty} 											%
	\newpage														%
}																%
																%
% -------------------------------------------------------------------------------------------------------------%
% Title Page / Document Class Definitions (Please Don't Play With This)			%
% -------------------------------------------------------------------------------------------------------------%
\date{}                                           	% Activate to display a given date or no date


% -------------------------------------------------------------------------------------------------------------
% DOCUMENT START (YOU CAN IGNORE EVERYTHING ABOVE HERE)					
% -------------------------------------------------------------------------------------------------------------
\begin{document}

% ---------------------------------------------------------------------------------------------------------------------------------------
% Title Page START: \grouptitlepage{doc number}{Version Number}{doc title}{group responsible for doc}		
% ---------------------------------------------------------------------------------------------------------------------------------------
\grouptitlepage
%Document Code Number (same as time reports)
{11}
%Document Version Number										
{0.5}
%Document Title		Dokumentmall							
{System Development Plan}
%Group Responsible For Document									
{(PG) Projekt Grupp: \\ Gregory Austin \\ Andreas Mårdén}	
% -------------------------------------------------------------------------------------------------------------
% Title Page END				
% -------------------------------------------------------------------------------------------------------------

\section{Inledning}
Detta dokument beskriver utvecklingsmodell och utvecklingsplan för {\color{red}{(Namn på produkt)-projektet}}. <--- Not: detta är direkt stulet {\color{red}{(Namn på produkt) (är)/(kommer bli)}} ett system för tidsrapportering för större projekt baserat på ``Baseblock system''. {\color{red}{(Namn på produkt)}} kommer utvecklas av studenter på kursen "Programvaruutveckling för stora projekt" vid Lunds Tekniska Högskola.

\section{Referensdokument}
{\color{red}{Referera här till (??????) Referens till handledningen kanske? Baseblock System?}}\\
{\color{red}{Referera till handledningen, CML SRS och SVVS... Tror jag...}}
{\color{red}{PUSP174210, PUSP174212, PUSP174213}}
\section{Utvecklingsplan}
Beroende på vad det är som refereras i föregående stycke så förändras innehållet här. Med antagande att det är handledningen som refereras: Skriv om något möte/dokument utgår (hittills inga)

\subsection*{Dokument}

\begin{tabular}{| l | l | l |}
\hline
\textbf{Svenska} & \textbf{Engelska} & \textbf{Förkortning} \\
\hline
\hline
 Konfigurationsenhetslista & Configuration Management List & CML \\
 \hline
Utvecklingsplan	 & Software Development Plan & SDP \\
\hline
Kravspecifikation & Software Requirements Specification & SRS \\
\hline
Testspecifikation & Software Verification and Validation Specification & SVVS \\
\hline
Testinstruktion & Software Verification and Validation Instruction & SVVI \\
\hline
Högnivådesign & Software Top Level Design Document & STLDD \\
\hline
Lågnivådesign & Software Detailed Design Document & SDDD \\
\hline
Testrapport & Software Verification and Validation Report & SVVR \\
\hline
Systemspecifikation & System Specification Document & SSD \\
\hline
Projektetsslutrapport & Project Final Report & PFR \\
\hline
\end{tabular}\\
%{\color{red}{Lägga till Konfigurationsenheteslista?}}

\section{Personalorganisation}
{\color{red}{NOTERA: Nämda dokument i nedanstående text kan ersättas med sina koder när dokumenttabellen i stycket ovan är klart}}
Följande roller definjeras i projektplanen: 
\begin{itemize}
\item Kund
\item Kvalitetsutvärderare
\item Sektionschef
\item Experter
\item Förändringskontroll
\item Projektledare
\item Systemansvarig
\item Utvecklare
\item Testare
\end{itemize}

\subsection*{Kund}
Kunden är den som ger projektgruppen dess uppdrag och är mottagare av resultatet. Kunden ska även godkänna den levererade produkten. Kunden har rätt att under projektets gång kunna gå in i projektet och göra en extern kvalitetsutvärdering. Denna innefattar rätten att kontrollera att projektet följer den planerade processen och att status gentemot den framtagna utvecklingsplanen stämmer. 
\begin{itemize}
\item Christin Lindholm är kund under detta projekt. 
\end{itemize}

\subsection*{Kvalitetsutvärderare}
Kvalitetsutvärderaren kommer att granska majoriteten av de dokument som projektet genererar. Vid formella granskningar kommer denne representera kunden och arbeta för att förbättra projektets kvalité, standard och för att kontrollera att utvecklingsmodellen följs. Kvalitetsutvärderaren har rätt att tillgå projektet och kontrollera dess status.
\begin{itemize}
\item Alma Orucevic-Alagic är kvalitetsutvärderare under detta projekt.
\end{itemize}

\subsection*{Sektionschef}
Sekstionschefen är projektgruppens högste chef och ska hjälpa projektgruppen med eventuella icke-tekniska problem som kan dyka upp under projektets gång. Dock skall alla problem i görligaste mån först tas upp med projektledargruppen, som står för kontakten med sektionschefen.
\begin{itemize}
\item Christin Lindholm är sektionschef för detta projekt.
\end{itemize}
		
	 
\subsection*{Experter}
% övernskommelse  -> överenskommelse ?
Experter finns tillgängliga för projektet för ingående frågor gällande Krav, Design och Test. Dessa experter är inte ingående i projektet men kan träffas efter övernskommelse.
\begin{itemize}
\item Alma Orucevic-Alagic och Anders Bruce är experter till detta projekt. %  (????????????)
\end{itemize}
	

\subsection*{Förändringskontrollgrupp}
% Systemansvariga -> systemansvariga ?
% förändringskontrollsgrupp -> förändringskontrollgrupp ?
% \item FKG består av Benjamin, Carl, Marlina, Andreas, Gregory ?
Under detta projekt kommer kombinationen av Systemansvariga och projektledare agera förändringskontrollsgrupp. Denna är ansvarig för konfigurationshanteringen. Huvudansvariga är de systemansvariga men projektledarna är inblandade för att kunna fatta beslut om ändringsåtgärder som kräver resurs- eller planeringsändringar.
\begin{itemize}
\item Förändringskontrollgruppen refereras hädanefter som FKG. 
\end{itemize}
		
\subsection*{Projektledargrupp}
Projektledargruppen har det övergripande administrativa ansvaret för projektet och är ytterst ansvariga för slutprodukten. De skall producera dokument som detaljerar: Tidsplan, Utvecklingsplan, Konfigurationsenheter, Projektets slutrapport och Systemets specifikation.
Projektledargruppen skall även ansvara för att övriga gruppmedlemmar har:
\begin{itemize}
\item nödvändig utbildning och  information
\item arbetsuppgifter
\item en jämn arbetsbelastning
\end{itemize}
Vidare skall projektledargruppen:	
\begin{itemize}	 
\item kontrollera att tidplanen följs och håller
\item ansvara för kontakterna med kunden, granskaren och sektionschefen
\item Sammankall till möten
\item Ansvara för dokumentbibliotekets tillgänglighet och organisation
\item Kontrollera och ansvara för projektmedlemmarnas tidsrapportering
\end{itemize}
Projektledare i detta projekt är Andreas Mårdén och Gregory Austin.	
\begin{itemize}
\item Projektledargruppen kommer hädanefter refereras till som PG.
\end{itemize}

\subsection*{Systemgrupp}
Systemgruppen består av de systemansvariga projektmedlemmarna. Dessa är ansvariga för att leda det tekniska arbetet och slutproduktens design. Systemgruppen skall tillsammans med utvecklingsgruppen producera dokument såsom Kravspecifikation, Högnivådesign och Lågnivådesign. De skall även ansvara för god konsistens mellan Kravspecifikationen och Testspecifikationen. De skall också ha hög förståelse för grundsystemet "Baseblock system" och den produkt som skapas.\\
{\color{red}{------Ansvar: (sammanskriv senare detta med resten av texten)}}  % förstår inte vad menas med det här
\begin{itemize}	 
\item Ansvara för gränssnitten mellan olika delar av systemet
\item Övervaka och styra all utveckling för att försäkra att systemets delar blir så likvärdiga som möjligt
\end{itemize}
% i detta projekt
Systemansvariga är Benjamin Holmqvist, Carl Rikner och Marlina Degirmenci. 
\begin{itemize}
\item Systemgruppen kommer hädanefter refereras till som SG.
\end{itemize}

\subsection*{Utvecklingsgrupp}
Utvecklingsgruppen kommer ta hand om utvecklingen av det system som skall levereras. Utvecklingsgruppen skall producera delkapitel för sin funtionalitet i dokumenten Kravspecifikation, Högnivådesign och Lågnivådesign. De är huvudansvariga för att utveckla funktionalitet enligt kravspecifikationen. Utvecklare i detta projekt är: Carl Gustavsson, Christian Shehadeh, Javier Poremski, Johannes Sunnanväder, Maurits Johansson, Richard Elvhammar, Sebastian Bergdahl, Simon Farre. \\
{\color{red}{------Unit-testa sin egen kod. }}
\begin{itemize}
\item Utvecklingsgruppen kommer hädanefter refereras till som UG.
\end{itemize}

\subsection*{Testgrupp}
Testgruppen har ansvaret för testningen av det utvecklade systemet. De skall producera dokumenten ``Testspecifikation'', ``Testinstruktioner'' samt en avslutande rapport för produktens testresultat. En utpekad testledare skall vidare ha ansvaret för konsistens mellan testspecifikationen och kravspecifikationen tillsammans med en utvald systemledare. Testgruppen består av: Emil Kristiansson, Erik Rosenström och Simon Plato.\\
{\color{red}{--------Producera monitorfiler för alla funktionstest (?)}}\\
{\color{red}{-------Föra versionshanterig av en massa appendix???}}
\begin{itemize}
\item  Testgruppen kommer hädanefter refereras till som TG.
\end{itemize}

\section{Tidplan}

{\color{red}{Jag har förslag på ett mall för en kalender vi kan sätta in för specifikt planering och vi kan hålla tabellerna till översiktsnivå, men vi kan diskutera det när vi har detaljerna}}\\
\begin{tabular}{| l | c | c | c | c |}
\hline
\textit{uppskattad} & \textbf{Fas 1} & \textbf{Fas 2} & \textbf{Fas 3} & \textbf{Fas 4}\\
\hline
\hline
\textbf{Start:} & v3 & v6 & v8 & v9 \\
\hline
\textbf{Stopp:} & v6 & v9 & v11 & v12 \\
\hline
 				& SDP & & & PFR \\
\textbf{Dokument:} & SRS & STLDD & SDDD & SSD \\
 				& SVVS & SVVI & & SVVR \\
\hline
\textbf{InfGran:} & v5 30/1 & v8 måndag & & \\
\hline
\textbf{FormGran:} & v6 10/2 & v8 torsdag & & \\
\hline
\textbf{OmGran:} & & v9 torsdag & & \\
\hline
\end{tabular}

\begin{tabular}{| l | c | c | c | c | c | c | c | c | c | c |}
\hline
\textit{uppskattad} & \textbf{v3} & \textbf{v4} & \textbf{v5} & \textbf{v6} & \textbf{v7} & \textbf{v8} & \textbf{v9} & \textbf{v10} & \textbf{v11} & \textbf{v12}\\
\hline
\textbf{Möte:} & & & & & & & & & & \\
\hline
\textbf{Arbetstid:} & & & & & & & & & & \\
\hline
\end{tabular}

\section{Standard och hjälpmedel}
Specifikationer av programhjälpmedel, tekniker och metoder

\subsection*{Programmeringsspråk som används:}
	\begin{itemize}
	\item LaTeX för dokumenten.
	\item Java för systemutveckling.
	\item SQL för att styra databasfunktioner.
	\item HTTP för implementera ett användargränssnitt.
	\end{itemize}

\subsection*{Mjukvara som används:}
	\begin{itemize}
	\item Discord för kommunikation i gruppen.
	\item Git för konfigurationsstyrning och dokumentbibliotek.
	\item Github.com som servrar till gruppens Git-arkiver och grenar.
	\item Apache Tomcat för att implementera servrar.
	\item MySQL för att implementera databas.
	\item Eclipse IDE som redan stödjer Apache Tomcat
		\begin{itemize}
		\item EGit för användning av Git funktioner i Eclipse.
		\item TeXlipse för att skapa LaTeX dokument i Eclipse.
		\item MySQL Connector/J för att hantera SQL i Eclipse.
		\item WST paket som innehåller en HTML editor för Eclipse. 
		\end{itemize}
	\end{itemize}

\subsection*{Design- och kodningsstandarder}
	\begin{itemize}
	\item Gruppens dokumentmall ska användas.
	\item "Designen följer SDL88" <---- Stulet och vad är SDL88 för något?
	\item Variabelnamn skall vara självförklarande och på engelska <---- också stulet men tycker jag att vi behåller
\end{itemize}



\section{Konfigurationshantering}
%{\color{red}{Måste läga till mer men det kommer räka tills vidare}} 
\subsection*{Översikt}
Projektet kommer följa ett arbetsflöde där konfigurationshanteringsystemet Git
integreras. Alla konfigurationsenheter som går i baseline samt alla dokument och systemenheter där utveckling pågår ska lagras i ett projektbibliotek. Mötesdokument kommer också finns tillgängligt i projektbiblioteket.
 
\subsection*{Projektbiblioteket}
\begin{itemize}
\item Projektbiblioteket implementeras av en Git-arkiv.
\item Arkiven kommer består av två delar: Ett konfigurationsbibliotek och ett mötesbibliotek.
\item Alla dokument och systemkomponenter som definieras i CML:n kommer lagras i konfigurationsbiblioteket.
\item Alla mötesprotokoll och agenda kommer lagras i mötesbiblioteket.
\item Det kommer finnas tre grenar av projektbibleoteket (dvs. master,
  $\alpha$ och $\beta$) under projektet.
\item Dokument- och systemenhetsversion nummer ändras bara om den läggs till
eller ändras i master-grenen.
\end{itemize}

\subsection*{Arbetsflödet}
\begin{itemize}

\item Master-grenen av projektbiblioteket kommer finnas tillgängligt via gruppens Github.
	\begin{itemize}
	\item FKG har ansvar för beslut om när master ska ändras.
	\item Den ska bara innehålla baseline dokument och baseline systemenheter.
	\item ``Hotfixes'' kan tillkomma om en konfigurationsenhet i baseline måste ändras.
	\end{itemize}

\item $\alpha$-gren innehåller konfigurationsenheterna som utvecklas.
	\begin{itemize}
	\item SG har ansvar för $\alpha$.
	\item När en konfigurationsenhet är i utvekling, gren från $\alpha$ skapas för
	att utveckla en specikifk del av enheten.
	\item Strukturen av gren(ar) från $\alpha$ bestäms av SG och UG efter behov.
	\item SG fattar beslut om när enheter ska integreras i $\beta$ för
	rekursivtestning. 
	\end{itemize}
	
\item $\beta$-grenen är en gren från master där rekursivtestning äger rum.
	\begin{itemize}
	\item TG har ansvar för $\beta$. 
	\item $\beta$ tar i mot enheter från $\alpha$-grenen. 
	\item TG fattar beslut om ett dokument eller en systemenhet ska till FKG för att integrera i master eller om den måste utvecklas mer. 
	\end{itemize}
	
\end{itemize}

\section{Regler}
Projektgruppen har via möten etablerat vissa regler som kommer påverka hur arbetet förs.
De gemensamma regler som vedertagits är: {\color{red}{vedertagits: stavning?}}
\begin{itemize}
\item Man kan då man skickar viktig info via Discord eller mejl begära "Ack". Dessa visar vilka som tagit del av informationen. Vid begärt 		  ack skall dessa inkomma inom 24 timmar.	
\item Arbete sker på arbetsdagar, inte helger, såvida inte annats beslutas gemensamt. Med arbetsdagar menas tiden mellan 8 och 16 Måndag till Fredag. Dock så är man välkommen att arbeta vilka tider man själv vill. Men ingen kan kräva andra att arbete utanför arbetsdagar.
\item Alla projektmedlemmar skall kontrollera sin mejl och discord dagligen. 
\item Alla projektmedlemmar har ett eget ansvar att hålla sig informerad och att informera.
\item Man meddelar i förhand om man inte kan komma på ett möte. Att meddela via ombud är fullt acceptabelt.
\item Inför formell granskning skall -alla- läsa -alla- dokument. % tog ut ``informell och''
\item Gruppen arbetstider är inte obligitoriska. Tidsplanering för specifika
dokument- eller funktionsutveckling ska prioritiseras.
\item Välja en person att ställa en fråga till experterna. 
\end{itemize}

\section{Uppföljning och kvalitetsutvärderingsprocess}

%Uppföljning och kvalitetsutvärderingsprocess
\begin{itemize}
\item Möten kommer ske två gånger om veckan där framsteg och problem diskuteras.
\item Att alla grupper kommer dela arbetsrum gör att samtliga kommer vara intimt medvetna om hur arbetet fortskrider ikring dem.
\item Om arbetet går snabbare än beräknat kommer faser och dokument tidigareläggas.
\item Om arbetet går långsammare än beräknat kommer vi hantera situationen beroende på dess natur.
\item Vid omgranskning/omarbete kommer vi sträva efter att gruppen så mycket som möjligt fortsätter med arbetet på de delar som är stadigast
\item Vid tidsbrist i programmeringen kommer SG och PG sättas in som programmerare
\item Vid tidsbrist i testningen (Något vi förutser) kommer utvecklare sättas in som testare
\item Projektets kvalité kommer försäkras genom de informella granskningar som kommer ske enligt schemat (minst en per fas) och genom testning
\item Denna testning kommer att utföras av UG (unit-tests) och TG (black-box testning)
\end{itemize}

\section{Riskanalys}

%Riskanalys:
Risker som kan drabba projektet innefattar: 
\begin{itemize}
\item Kollaps av kommunikationssystem: {\color{red}{Vill diskutera}}
\end{itemize}

\begin{tabular}{| l | c | c | l |}
\hline
	& \textbf{Sannolikhet} & \textbf{Möjlig Effekt} & \textbf{Komentar} \\
\hline
\textbf{Konflikter} & Hög & Låg & Konflikter -bör- hända i någon form. \\
\hline
\textbf{Sjukdomsfrånvaro} & Medel & Medel & Allvarligare sjukdom känns osannolikt. \\
\hline
\textbf{Större omarbete} & Hög & Låg & Planeringen förutsätter att detta sker. \\
\hline
\textbf{Arbetskraftsfrånvaro} & Hög & Hög & Vår mest bekymmersamma risk. \\
\hline
\textbf{Orealistiskt schema} & Låg & Hög & Låg risk då uppgiften har anpassats för kursen.\\
 & & & {\color{red}{kommer inte risken från oss mer än kursen?}} \\
\hline
\textbf{Utvecklar efter fel krav} & Medel & Medel & Med god STLDD och SG kan detta undvikas. \\
{\color{red}{Funkioner}} & & &  \\
{\color{red}{Interface}} & & &  \\
{\color{red}{``Gold plating''}} & & &  \\
\hline
\textbf{Låg produkt prestanda} & Låg & Låg &  \\
\hline
\textbf{Brist på {\color{red}{kunskaper}}} & Låg & Hög & Kursen är anpassad för utbildningen. \\

\hline
\textbf{Oväntat komplexitet} & Låg & Hög & Kursen är anpassad för utbildningen.  \\
\hline
\textbf{Personalslitage(?)} & Medel & Medel &  \\
\hline

\end{tabular}
%	
%	Kollaps av kommunikationssystem:
%		Sannolikhet: Låg	Möjlig effekt: medel-hög 
%		* Vi räknar knappast med att github eller discord förlorar all vår data.
%	
%	Konflikter:				
%		Sannolikhet: Hög	Möjlig effekt: Låg
%		* Konflikter -bör- hända i någon form. I alla fall som resultat av misskommunikation.
%	
%	Sjukdomsfrånvaro:
%		Sannolikhet: Medel	Möjlig effekt:	Medel
%		* Då allvarligare sjukdom känns osannolikt så orsakar sjukdom nog på sin höjd förseningar.
%
%	Större omarbeten efter granskningar
%		Sannolikhet: Hög	Möjlig effekt: Låg	
%		* Då planeringen förutsätter att detta sker så är riskens troliga effekt låg.
%
%Från boken:
%	Arbetskraftsfrånvaro 			Hög, Hög (Vår mest bekymmersamma risk)
%	Orealistiskt schema			Låg, Hög (låg risk då uppgiften har anpassats för kursen)
%	Utvecklar fel funkioner			Medel, Medel (Med god STLDD och leaderskap från SG så kan detta undvikas)
%	Utvecklar fel användarinterface		medel, låg (Se ovan)
%	"Gold plating"				låg, medel 
%Från boken 2:
%	Risk för låg prestanda: Låg, låg 
%	Brist på rätt utbildad personal: Låg, hög (låg risk då kursen är anpassad för utbildningen)
%	Oväntat hög komplexitet av systemet: låg, hög (igen, då uppgiften är anpassad för oss)
%	Personalslitage(?): medel, medel
















\end{document}